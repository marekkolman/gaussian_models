\documentclass[11pt,a4paper]{article}
\usepackage[left=1.5cm,top=3.5cm,bottom=3cm,right=1.5cm]{geometry}
%\usepackage[T1]{fontenc}
%\usepackage{yfonts}
\usepackage{amsfonts}
\usepackage{enumerate}
\usepackage{amsmath}
\usepackage{amssymb} % for \nrightarrow
\usepackage[square,sort, comma, numbers]{natbib}
\usepackage{graphicx}
\usepackage{epstopdf}
\usepackage{todonotes}
\usepackage{xcolor}
\usepackage{hyperref}
\hypersetup{colorlinks=true, linkcolor = blue, citecolor = red}
\usepackage{theorem} % package pro nastaveni zacatku theoremu na novem radku
\theoremstyle{break} % vynuceni noveho radku u theoremu

\newtheorem{example}{Example}[section]
\newtheorem{assumption}{Assumption}[section]
\newtheorem{definition}{Definition}[section]
\newtheorem{theorem}{Theorem}[section]
\newtheorem{lemma}{Lemma}[section]
\newtheorem{proposition}{Proposition}[section]
\newtheorem{corollary}{Corollary}[section]
\newenvironment{proof}{\noindent \textit{Proof.}}{\hfill$\square$}
\numberwithin{equation}{section}
\graphicspath{{img}}
\usepackage{tcolorbox} % command to make a box
\newtcolorbox[auto counter, number within = section]{pabox}[2][]{}
\begin{document}
	
\begin{titlepage}
\begin{center}
\textsc{\LARGE Multifactor Gaussian models}\\[0.2cm] % Name of your university/college
Marek Kolman \\ University of Economics, Prague
\end{center}
\vspace{25pt}
\begin{abstract}
	This text serves as a simple technical document for multifactor Gaussian IR models implemented in the attached script. The model is fully based on \citep[Section 12]{piterbargII}. Attention is paid especially to a 2-factor Gaussian model. 
	\\ \\ \noindent Related python code repository can be found at:
\begin{pabox}{Code repository}
	\begin{center}
		\url{https://github.com/marekkolman/gaussian_models}
	\end{center}
\end{pabox}
\noindent The code implements the two-factor model and also contains a calibration data set.
\end{abstract}
\end{titlepage}

\section{Multifactor Gaussian model}
In a general $d$-factor \textit{Gaussian} model, the $T$--bond price $P$ follows
\begin{equation}\label{eq:bond_price_gaussian_2f}
  dP(t,T)/P(t,T) = r(t)dt - {\sigma _P}{(t,T)^{\rm{T}}}dW(t),
\end{equation}
where ${\sigma _P}(t,T), dW(t)$ are $d$-dimensional column vectors and $W$ is a $\mathbb{Q}$--Wiener. In terms of instantaneous forward rates, the above can equivalently be written in terms of instantaneous forward rates
\begin{equation*}
  df(t,T) = {\sigma _f}{(t,T)^{\rm{T}}}{\sigma _P}(t,T)dt + {\sigma _f}{(t,T)^{\rm{T}}}dW(t).
\end{equation*}
To prevent arbitrage, HJM condition must hold and we thus have based on the HJM drift restriction
\begin{equation*}
  df(t,T) = {\sigma _f}{(t,T)^{\rm{T}}}\int\limits_t^T {{\sigma _f}(t,u)du} dt + {\sigma _f}{(t,T)^{\rm{T}}}dW(t).
\end{equation*}
To ensure that this model is Markovian, the following 'separability' condition must necessarily hold.

\begin{theorem}[Forward rate volatility separability condition]
If the forward-rate volatility ${\sigma _f}(t,T)$ is separable such that
\begin{equation*}
  {\sigma _f}(t,T) = g(t)h(T),
\end{equation*}
where $g$ is a $d \times d$ deterministic matrix-valued function and $h$ is a $d$--dimensional vector-valued function,
then $f$ is Markovian and
\begin{equation*}
  f(t,T) = f(0,T) + \Omega (t,T) + h{(T)^{\rm{T}}}z(t),
\end{equation*}
where
\begin{eqnarray*}
dz(t) &=& g{(t)^{\rm{T}}}dW(t), \text{ given } z(0) = 0,\\
\Omega (t,T) &=& h{(T)^{\rm{T}}}\int\limits_0^t {g{{(s)}^{\rm{T}}}g(s)\int\limits_s^T {h(u)du} ds}.
\end{eqnarray*}
In particular we also have
\begin{equation*}
  r(t) = f(0,t) + \Omega (t,t) + h{(t)^{\rm{T}}}z(t).
\end{equation*}
\end{theorem}
Note that with this separability condition in effect, the HJM drift restriction implies
\begin{equation}\label{eq:sigmaP_gaussian2f}
  {\sigma _P}(t,T) = \int\limits_t^T {{\sigma _f}(t,u)du}  = g(t)\int\limits_t^T {h(u)du}.
\end{equation}
To formulate a mathematically tractable and economically meaningful model, it is best to define a particular functional form of mean reversion. One useful choice is to set
\begin{equation*}
  H(t) = {\rm{diag}}(h(t)).
\end{equation*}
If elements of $h$ are non-zero, then $H$ is invertible and we can define a 'mean-reversion' matrix
\begin{equation}\label{eq:MR_mat_gaussian}
  \varkappa(t)  =  - \frac{{dH(t)}}{{dt}}H{(t)^{ - 1}}.
\end{equation}
It is also very useful to formulate the model in terms of state variables $x$ such that
\begin{eqnarray}
dx(t) &=& \left( {y(t){\bf{1}} - \varkappa (t)x(t)} \right)dt + {\left( {g(t)H(t)} \right)^{\rm{T}}}dW(t)\nonumber \\
dy(t)/dt &=& H(t)g{(t)^{\rm{T}}}g(t)H(t) - \varkappa (t)y(t) - y(t)\varkappa (t).\label{eq:dx_dy_gaussian2f}
\end{eqnarray}
Here $x$ is a $d$-valued vector process with $x(0) = {\bf{0}}$ and $y(0)$ is a $d \times d$ matrix-valued deterministic function such that $y(0)={\bf{0}}$. The differential equations for can be solved for $x, y$ and this reads
\begin{eqnarray}
x(t) &=& H(t)\int\limits_0^t {g{{(s)}^{\rm{T}}}g(s)\left( {\int\limits_s^T {h(u)du} } \right)ds}  + H(t)z(t)\nonumber \\
y(t) &=& H(t)\left( {\int\limits_0^t {g{{(u)}^{\rm{T}}}g(u)du} } \right)H(t).\label{eq:y_gaussian2f}
\end{eqnarray}
This formulation yields
\begin{equation*}
f(t,T) = f(0,T) + M{(t,T)^{\rm{T}}}\left( {x(t) + y(t)\int\limits_t^T {M(t,u)du} } \right)
\end{equation*}
with
\begin{equation*}
M(t,T) = H(T)H{(t)^{ - 1}}{\bf{1}}.
\end{equation*}
Notice that this results into a convenient representation of $r$ in terms of the state variables $x$ as
\begin{equation*}
  r(t) = f(t,t) = f(0,t) + \sum\limits_{i = 1}^d {{x_i}(t)}.
\end{equation*}
Furthermore, letting 
\begin{equation*}
  G(t,T) = \int\limits_t^T {M(t,u)du} ,
\end{equation*}
we can define a useful bond reconstitution formula. 
\begin{definition}[Bond reconstitution formula in Gaussian models]
Given the above settings in place, the bond price can be computed as
\begin{equation*}
  P(t,T)\mathop  = \limits^\Delta  P(t,T,x) = \frac{{P(0,T)}}{{P(0,t)}}\exp \left( { - G{{(t,T)}^{\rm{T}}}x(t) - \frac{1}{2}G{{(t,T)}^{\rm{T}}}y(t)G(t,T)} \right).
\end{equation*}
\end{definition}

\subsection{Two-factor Gaussian model}
Of a particular interest is a two factor model with \textit{constant} coefficients. Such a model will be fully determined by a mean reversion speeds $\varkappa_1, \varkappa_2$ and a diffusion matrix $\sigma(t)=\sigma$. Let us define $g$ and $h$ as
\begin{eqnarray}
g(t) &=& \left( {\begin{array}{*{20}{c}}
{{\sigma _{11}}(t){e^{\int\limits_0^t {{\varkappa _1}(u)du} }}}&{{\sigma _{12}}(t){e^{\int\limits_0^t {{\varkappa _2}(u)du} }}}\\
{{\sigma _{21}}(t){e^{\int\limits_0^t {{\varkappa _1}(u)du} }}}&{{\sigma _{22}}(t){e^{\int\limits_0^t {{\varkappa _2}(u)du} }}}
\end{array}} \right) \equiv \left( {\begin{array}{*{20}{c}}
{{\sigma _{11}}{e^{{\varkappa _1}t}}}&{{\sigma _{12}}{e^{{\varkappa _2}t}}}\\
{{\sigma _{21}}{e^{{\varkappa _1}t}}}&{{\sigma _{22}}{e^{{\varkappa _2}t}}}
\end{array}} \right)\\
h(t) &=& \left( {\begin{array}{*{20}{c}}
{{e^{ - \int\limits_0^t {{\varkappa _1}(u)du} }}}\\
{{e^{ - \int\limits_0^t {{\varkappa _2}(u)du} }}}
\end{array}} \right) \equiv \left( {\begin{array}{*{20}{c}}
{{e^{ - {\varkappa _1}t}}}\\
{{e^{ - {\varkappa _2}t}}}
\end{array}} \right).
\end{eqnarray}
Having a look at the diffusion coefficient (matrix) of $x$ in \eqref{eq:dx_dy_gaussian2f} gives
\begin{equation*}
  g(t){\rm{diag(}}h(t)) = g(t)H(t) = \left( {\begin{array}{*{20}{c}}
{{\sigma _{11}}}&{{\sigma _{12}}}\\
{{\sigma _{21}}}&{{\sigma _{22}}}
\end{array}} \right) = \sigma (t) \equiv \sigma.
\end{equation*}
Furthermore following \eqref{eq:MR_mat_gaussian} we also get a mean reversion matrix
\begin{equation}
  \varkappa (t) = \varkappa  = \left( {\begin{array}{*{20}{c}}
{{\varkappa _1}}&0\\
0&{{\varkappa _2}}
\end{array}} \right).
\end{equation}
Under this setup, the state variables $x = {({x_1},{x_2})^{\rm{T}}}$ in \eqref{eq:dx_dy_gaussian2f} follow 
\begin{eqnarray*}
\left( {\begin{array}{*{20}{c}}
{d{x_1}(t)}\\
{d{x_2}(t)}
\end{array}} \right) &=& \left( {\begin{array}{*{20}{c}}
{{y_{11}}(t) + {y_{12}}(t) - {\varkappa _1}{x_1}(t)}\\
{{y_{21}}(t) + {y_{22}}(t) - {\varkappa _2}{x_2}(t)}
\end{array}} \right)dt + {\sigma ^{\rm{T}}}dW(t)\\
 &=& \left( {\begin{array}{*{20}{c}}
{{v_1}(t) - {\varkappa _1}{x_1}(t)}\\
{{v_2}(t) - {\varkappa _2}{x_2}(t)}
\end{array}} \right)dt + \left( {\begin{array}{*{20}{c}}
{{\sigma _{11}}}&{{\sigma _{21}}}\\
{{\sigma _{12}}}&{{\sigma _{22}}}
\end{array}} \right)\left( {\begin{array}{*{20}{c}}
{d{W_1}(t)}\\
{d{W_2}(t)}
\end{array}} \right),
\end{eqnarray*}
where ${v_1}(t) = {y_{11}}(t) + {y_{12}}(t),{v_2}(t) = {y_{21}}(t) + {y_{22}}(t)$.
It is convenient to write the above as two differentials
\begin{eqnarray*}
d{x_1}(t) &=& \left( {{v_1}(t) - {\varkappa _1}{x_1}(t)} \right)dt + {\sigma _{11}}d{W_1}(t) + {\sigma _{21}}d{W_2}(t)\\
d{x_2}(t) &=& \left( {{v_2}(t) - {\varkappa _2}{x_2}(t)} \right)dt + {\sigma _{12}}d{W_1}(t) + {\sigma _{22}}d{W_2}(t).
\end{eqnarray*}
To efficiently use $y(t)$ it is useful to borrow from \eqref{eq:y_gaussian2f} and analytically evaluate the integral 
\begin{equation*}
\bar g(t) = \int\limits_0^t {g{{(u)}^{\rm{T}}}g(u)du}  = \left( {\begin{array}{*{20}{c}}
{\frac{{({e^{2{\varkappa_1}t}} - 1)(\sigma _{11}^2 + \sigma _{21}^2)}}{{2{\varkappa_1}}}}&{\frac{{({e^{({\varkappa_1} + {\varkappa_2})t}} - 1)({\sigma _{11}}{\sigma _{12}} + {\sigma _{21}}{\sigma _{22}})}}{{{\varkappa_1} + {\varkappa_2}}}}\\
{\frac{{({e^{({\varkappa_1} + {\varkappa_2})t}} - 1)({\sigma _{11}}{\sigma _{12}} + {\sigma _{21}}{\sigma _{22}})}}{{{\varkappa_1} + {\varkappa_2}}}}&{\frac{{({e^{2{\varkappa_2}t}} - 1)(\sigma _{12}^2 + \sigma _{22}^2)}}{{2{\varkappa_2}}}}
\end{array}} \right).
\end{equation*}
Then, we can calculate inexpensive matrix multiplication $y(t) = H(t)\bar g(t)H(t)$.

It is also useful to analytically compute $M$ and $G$ which gives
\begin{equation*}
  M(t,T) = \left( {\begin{array}{*{20}{c}}
{{e^{ - {\varkappa _1}(T - t)}}}\\
{{e^{ - {\varkappa _2}(T - t)}}}
\end{array}} \right),G(t,T) = \left( {\begin{array}{*{20}{c}}
{\frac{{1 - {e^{ - {\varkappa _1}(T - t)}}}}{{{\varkappa _1}}}}\\
{\frac{{1 - {e^{ - {\varkappa _2}(T - t)}}}}{{{\varkappa _2}}}}
\end{array}} \right).
\end{equation*}

For practical calculations, we often prefer to use $\mathbb{Q}_T$--measure such that
\begin{equation*}
  V(t) = {\mathbb{E}^\mathbb{Q}}\left[ {\left. {{e^{ - \int\limits_t^T {r(u)du} }}V(x(T),T)} \right|{\mathcal F}(t)} \right] = P(t,T){\mathbb{E}^{{\mathbb{Q}_T}}}\left[ {\left. {V(x(T),T)} \right|{\mathcal F}(t)} \right].
\end{equation*}
To do this, we need to use a $T$--bond $P(\cdot, T)$ as a numeraire asset. A bond price has the dynamics \eqref{eq:bond_price_gaussian_2f} and $\sigma_P$ is necessary. We can compute $\sigma_P$ analytically and it gives
\begin{equation*}
  {\sigma _P}(t,T) = \left( {\begin{array}{*{20}{c}}
{\frac{{1 - {e^{ - {\varkappa_1}(T - t)}}}}{{{\varkappa_1}}}{\sigma _{11}} + \frac{{1 - {e^{ - {\varkappa_2}(T - t)}}}}{{{\varkappa_2}}}{\sigma _{12}}}\\
{\frac{{1 - {e^{ - {\varkappa_1}(T - t)}}}}{{{\varkappa_1}}}{\sigma _{21}} + \frac{{1 - {e^{ - {\varkappa_2}(T - t)}}}}{{{\varkappa_2}}}{\sigma _{22}}}
\end{array}} \right) = \left( {\begin{array}{*{20}{c}}
{{G_1}(t,T){\sigma _{11}} + {G_2}(t,T){\sigma _{12}}}\\
{{G_1}(t,T){\sigma _{21}} + {G_2}(t,T){\sigma _{22}}}
\end{array}} \right).
\end{equation*}
Thus, we have
\begin{eqnarray*}
dP(t,T)/P(t,T) &=& r(t)dt - {\sigma _P}{(t,T)^{\rm{T}}}dW(t)\\
 &=& r(t)dt - {\sigma _{P,1}}(t,T)d{W_1}(t) - {\sigma _{P,2}}(t,T)d{W_2}(t)\\
 &=& r(t)dt - \left( {{G_1}(t,T){\sigma _{11}} + {G_2}(t,T){\sigma _{12}}} \right)d{W_1}(t) - \left( {{G_1}(t,T){\sigma _{21}} + {G_2}(t,T){\sigma _{22}}} \right)d{W_2}(t).
\end{eqnarray*}
Knowing the diffusion term of the Numeraire $P$, we can use Girsanov to switch to $\mathbb{Q}$ to $\mathbb{Q}_T$, i.e.
\begin{equation*}
  d{W^{{\mathbb{Q}_T}}}(t) = dW(t) + {\sigma _P}(t,T)dt.
\end{equation*}
After substituting for $dW(t)$, this means for $dx$ that
\begin{eqnarray*}
d{x_1}(t) &=& \left( {v_1^T(t) - {\varkappa_1}{x_1}(t)} \right)dt + {\sigma _{11}}dW_1^{{\mathbb{Q}_T}}(t) + {\sigma _{21}}dW_2^{{\mathbb{Q}_T}}(t)\\
d{x_2}(t) &=& \left( {v_2^T(t) - {\varkappa_2}{x_2}(t)} \right)dt + {\sigma _{12}}dW_1^{{\mathbb{Q}_T}}(t) + {\sigma _{22}}dW_2^{{\mathbb{Q}_T}}(t),
\end{eqnarray*}
where 
\begin{eqnarray*}
v_1^T(t) &=& {v_1}(t) - {\sigma _{11}}{\sigma _{P,1}}(t,T) - {\sigma _{21}}{\sigma _{P,2}}(t,T)\\
v_2^T(t) &=& {v_2}(t) - {\sigma _{12}}{\sigma _{P,1}}(t,T) - {\sigma _{22}}{\sigma _{P,2}}(t,T).
\end{eqnarray*}

Since option pricing is based on the (joint) distribution of the state variables $x_1, x_2$, under $\mathbb{Q}_T$, we note that 
\begin{eqnarray*}
{\mathbb{E}^{{\mathbb{Q}_T}}}[{x_i}(T)] &=& \int\limits_0^T {{e^{ - (T - u){\varkappa_i}}}v_i^T(u)du}  = 0\\
{\mathbb{V}^{{\mathbb{Q}_T}}}[{x_i}(T)] &=& \int\limits_0^T {{e^{ - 2(T - u){\varkappa_i}}}(\sigma _{1i}^2 + \sigma _{2i}^2)du}  = (\sigma _{1i}^2 + \sigma _{2i}^2)\frac{{1 - {e^{ - 2{\varkappa_i}T}}}}{{2{\varkappa_i}}}\\
\text{Cov}^{{\mathbb{Q}_T}}[{x_1}(T),{x_2}(T)] &=& \int\limits_0^T {{e^{ - (T - u)({\varkappa_1} + {\varkappa_2})}}({\sigma _{11}}{\sigma _{12}} + {\sigma _{21}}{\sigma _{22}})du}  = ({\sigma _{11}}{\sigma _{12}} + {\sigma _{21}}{\sigma _{22}})\frac{{1 - {e^{ - ({\varkappa_1} + {\varkappa_2})T}}}}{{{\varkappa_1} + {\varkappa_2}}}.
\end{eqnarray*}
Since $x_1, x_2$ are jointly normal, based on the properties of conditional normal variables we can obtain the $x_2$--conditional moments
\begin{eqnarray*}
  {\mathbb{E}^{{\mathbb{Q}_T}}}\left[ {\left. {{x_1}(T)} \right|{x_2}(T) = {x_2}} \right] &=& {\mathbb{E}^{{\mathbb{Q}_T}}}\left[ {{x_1}(T)} \right] + \frac{{\text{Cov}^{{\mathbb{Q}_T}}[{x_1}(T),{x_2}(T)]}}{{{\mathbb{V}^{{\mathbb{Q}_T}}}[{x_2}(T)]}}({x_2} - {\mathbb{E}^{{\mathbb{Q}_T}}}\left[ {{x_2}(T)} \right])\\
   &=& \frac{{{\rm{Co}}{{\rm{v}}^{{\mathbb{Q}_T}}}[{x_1}(T),{x_2}(T)]}}{{{\mathbb{V}^{{\mathbb{Q}_T}}}[{x_2}(T)]}}{x_2} = {\mu _1}(T,{x_2})\\
   {\mathbb{V}^{{\mathbb{Q}_T}}}\left[ {\left. {{x_1}(T)} \right|{x_2}(T) = {x_2}} \right] &=& {\mathbb{V}^{{\mathbb{Q}_T}}}[{x_1}(T)] - \frac{{{{\left( {\text{Cov}^{{\mathbb{Q}_T}}[{x_1}(T),{x_2}(T)]} \right)}^2}}}{{{\mathbb{V}^{{\mathbb{Q}_T}}}[{x_2}(T)]}} = s_1^2(T,{x_2})
\end{eqnarray*}
\subsection{Two-factor Gaussian model: bond option}
Assume a put option $p$ expiring at $T$ on a $s$--zero-bond $P(\cdot,s)$. Such an option has a payoff
\begin{equation*}
  p(T;T,s,K;{x_1}(T),{x_2}(T)) = \max \left[ {K - P(T,s;{x_1}(T),{x_2}(T)),0} \right].
\end{equation*}
In the two-factor Gaussian model bond price $P$ is a deterministic function of $x$:
\begin{equation}\label{eq:bond_price_gaussian2f}
  P(T,s, x(T)) = \frac{{P(0,s)}}{{P(0,T)}}\exp \left( { - {G_1}(T,s){x_1}(T) - {G_2}(T,s){x_2}(T) + A(T,s)} \right).
\end{equation}
At time $t=0$, under the measure $\mathbb{Q}_T$ the (put) bond option reads
\begin{eqnarray*}
p(0;T,s,K) &=& P(0,T)\int\limits_{ - \infty }^\infty  {{\mathbb{E}^{{\mathbb{Q}_T}}}\left[ {p(T;T,s,K;{x_1}(T),{x_2}(T))|{x_2}(T) = {x_2}} \right]{f_{{x_2}}}({x_2})d{x_2}} \\
 &=& P(0,T)\int\limits_{ - \infty }^\infty  {{\mathbb{E}^{{\mathbb{Q}_T}}}{{\left[ {K - \frac{{P(0,s)}}{{P(0,T)}}\exp \left( { - {G_1}(T,s){x_1}(T) - {G_2}(T,s){x_2} + A(T,s)} \right)} \right]}^ + }{f_{{x_2}}}({x_2})d{x_2}} \\
 &=& \int\limits_{ - \infty }^\infty  {{\mathbb{E}^{{\mathbb{Q}_T}}}{{\left[ {KP(0,T) - P(0,s)\exp \left( { - {G_1}(T,s){x_1}(T) - {G_2}(T,s){x_2} + A(T,s)} \right)} \right]}^ + }{f_{{x_2}}}({x_2})d{x_2}} \\
 &=& \int\limits_{ - \infty }^\infty  {{\mathbb{E}^{{\mathbb{Q}_T}}}{{\left[ {KP(0,T) - P(0,s)\exp \left( { - {G_1}(T,s){x_1}(T)} \right)\exp \left( { - {G_2}(T,s){x_2} + A(T,s)} \right)} \right]}^ + }{f_{{x_2}}}({x_2})d{x_2}} \\
 &=& \int\limits_{ - \infty }^\infty  {P(0,s)\exp \left( { - {G_2}(T,s){x_2} + A(T,s)} \right){\mathbb{E}^{{\mathbb{Q}_T}}}{{\left[ {{K^*} - \exp \left( { - {G_1}(T,s){x_1}(T)} \right)} \right]}^ + }{f_{{x_2}}}({x_2})d{x_2}}, 
\end{eqnarray*}
where
\begin{equation*}
  {K^*} = \frac{{P(0,T)}}{{P(0,s)}}\exp \left( {{G_2}(T,s){x_2} - A(T,s)} \right)K.
\end{equation*}
Thus, conditionally on $x_2(T)=x_2$ the bond option value at time $t=0$ is
\begin{equation}\label{eq:conditional_put_gaussian2f}
  p(0;T,s,K|{x_2}(T) = {x_2}) = P(0,s)\exp \left( { - {G_2}(T,s){x_2} + A(T,s)} \right){\mathbb{E}^{{\mathbb{Q}_T}}}{\left[ {{K^*} - \exp \left( { - {G_1}(T,s){x_1}(T)} \right)} \right]^ + }.
\end{equation}
The equation \eqref{eq:conditional_put_gaussian2f} is actually a Black-Scholes-like problem and therefore
\begin{eqnarray*}
p(0;T,s,K|{x_2}(T) = {x_2}) &=& P(0,s)\exp \left( { - {G_2}(T,s){x_2} + A(T,s)} \right)\left( {{K^*}N( - {d_ - }) - {e^{\Omega (T,s,{x_2})}}N( - {d_ + })} \right)\\
{d_ \pm } &=& \frac{{\Omega (T,s,{x_2}) - \ln {K^*} \pm \frac{1}{2}G_1^2(T,s)s_1^2(T,{x_2})}}{{{G_1}(T,s){s_1}(T,{x_2})}}\\
\Omega (T,s,{x_2}) &=&  - {\mu _1}(T,{x_2}){G_1}(T,s) + \frac{1}{2}G_1^2(T,s)s_1^2(T,{x_2}).
\end{eqnarray*}
For completeness we note that an analogous $x_2$--conditional call option price would be
\begin{equation*}
  c(0;T,s,K|{x_2}(T) = {x_2}) = P(0,s)\exp \left( { - {G_2}(T,s){x_2} + A(T,s)} \right)\left( {{e^{\Omega (T,s,{x_2})}}N({d_ + }) - {K^*}N({d_ - })} \right).
\end{equation*}
\subsection{Two-factor Gaussian model: Jamshidian decomposition}
A payer swaption $V$ with expiry $T$ has the payoff 
\begin{equation*}
  V({T_0}) = {\left( {1 - P({T_0},{T_N}) - K\sum\limits_{i = 1}^N {{\tau _i}P({T_0},{T_i})} } \right)^ + }.
\end{equation*}
The well known problem is that a direct swaption formula isn't available but bond option is so ideally we would like to leverage the knowledge of the bond option formula established above. This is possible but also not directly because the payoff can't be trivially decomposed into smaller units, only after a workaround. This workaround where a swaption payoff is decomposed into a combination of bond options is known as Jamshidian's trick and here we use a two-dimensional version.

We know that according to \eqref{eq:bond_price_gaussian2f} bond price $P$ is a decreasing function of $x$ (as $G$ is always positive). Thus conditionally on $x_2(T_0)=x_2$, the bond option only pays out if ${{x_1}(T) > x_1^*({x_2})}$. We can therefore write the payoff as 
\begin{equation*}
  V({T_0},{x_2}) = \left( {1 - P({T_0},{T_N};{x_1}(T),{x_2}) - K\sum\limits_{i = 1}^N {{\tau _i}P({T_0},{T_i},{x_1}(T),{x_2})} } \right){{\bf{1}}_{\{ {x_1}(T) > x_1^*({x_2})\} }},
\end{equation*}
where $x_1^*({x_2})$ is the breakeven value of $x_1(T)$ (for a given fixed $x_2$) that makes the swap in the swaption zero valued at $T_0$. Fixing $x_2$, $x_1^*({x_2})$ thus solves
\begin{equation*}
  1 - P({T_0},{T_N};x_1^*({x_2}),{x_2}) - K\sum\limits_{i = 1}^N {{\tau _i}P({T_0},{T_i},x_1^*({x_2}),{x_2})}  = 0.
\end{equation*}
Any univariate root-search algorithm can be used to solve this equation. Once $x_1^*({x_2})$ has been found, pseudo-strikes $K_i$ can be set up such that 
\begin{equation*}
  {K_i}({x_2}) = P({T_0},{T_i},x_1^*({x_2}),{x_2}),i = 1,...,N, 
\end{equation*}
and the above equality can be rewritten as
\begin{equation*}
  1 - {K_N}({x_2}) - K\sum\limits_{i = 1}^N {{\tau _i}{K_i}({x_2})}  = 0 \Rightarrow 1 = {K_N}({x_2}) + K\sum\limits_{i = 1}^N {{\tau _i}{K_i}({x_2})}.
\end{equation*}
In terms of $K_i$ we can write the swaption payoff as
\small
\begin{eqnarray*}
V({T_0},{x_2}) &=& \left( {{K_N}({x_2}) + K\sum\limits_{i = 1}^N {{\tau _i}{K_i}({x_2})}  - P({T_0},{T_N};{x_1}(T),{x_2}) - K\sum\limits_{i = 1}^N {{\tau _i}P({T_0},{T_i};{x_1}(T),{x_2})} } \right){{\bf{1}}_{\{ {x_1}(T) > x_1^*({x_2})\} }}\\
 &=& \left( {{K_N}({x_2}) - P({T_0},{T_N};{x_1}(T),{x_2})} \right){{\bf{1}}_{\{ {x_1}(T) > x_1^*({x_2})\} }} \\
&+& K\sum\limits_{i = 1}^N {{\tau _i}\left( {{K_i}({x_2}) - P({T_0},{T_i};{x_1}(T),{x_2})} \right){{\bf{1}}_{\{ {x_1}(T) > x_1^*({x_2})\} }}} \\
 &=& {\left( {{K_N}({x_2}) - P({T_0},{T_N};{x_1}(T),{x_2})} \right)^ + } + K\sum\limits_{i = 1}^N {{\tau _i}{{\left( {{K_i}({x_2}) - P({T_0},{T_i};{x_1}(T),{x_2})} \right)}^ + }}.
\end{eqnarray*}
\normalsize
This means a payer swaption can be written as a portfolio of bond put options
\begin{equation*}
  V(0,{x_2}) = {p_N}(0,{x_2}) + K\sum\limits_{i = 1}^N {{\tau _i}{p_i}(0,{x_2})},
\end{equation*}
where
\begin{equation*}
  {p_i}(0,{x_2}) = p(0;{T_0},{T_i},{K_i}|{x_2}(T_0) = {x_2}).
\end{equation*}
The unconditional payer swaption formula can then be obtained by integrating over the density of $x_2$, thus
\begin{eqnarray}
V(0) &=& \int\limits_{ - \infty }^\infty  {\left( {{p_N}(0,{x_2}) + K\sum\limits_{i = 1}^N {{\tau _i}{p_i}(0,{x_2})} } \right){f_{{x_2}}}({x_2})d{x_2}} \nonumber \nonumber \\
 &=& \int\limits_{ - \infty }^\infty  {\frac{{{p_N}(0,{x_2}) + K\sum\limits_{i = 1}^N {{\tau _i}{p_i}(0,{x_2})} }}{{\sqrt {{\mathbb{V}^{{\mathbb{Q}_{T_0}}}}[{x_2}(T_0)]} }}{f_N}\left( {\frac{{{x_2} - {\mathbb{E}^{{\mathbb{Q}_{T_0}}}}[{x_2}(T_0)]}}{{\sqrt {{\mathbb{V}^{{\mathbb{Q}_{T_0}}}}[{x_2}(T_0)]} }}} \right)d{x_2}},\label{eq:swaption_formula_gaussian2f}
\end{eqnarray}
where $f_{x_2}$ is the density for $x_2(T_0)$, i.e. a normal density for a variable with mean ${{\mathbb{E}^{{\mathbb{Q}_{T_0}}}}[{x_2}(T)]}$ and variance $\mathbb{V}{^{{\mathbb{Q}_{T_0}}}[{x_2}(T_0)]}$ and $f_N$ is the standard normal density.

\subsection{Approximating swaption formula for Gaussian models}
The swaption formula \eqref{eq:swaption_formula_gaussian2f} can be evaluated but numerically, involving a root search within an integral computation.\footnote{We integrate over various values of $x_2$ and at each value $x_2$ we are finding a critical value $x_1^*({x_2})$ by a root-search.} This is very inconvenient and technically too demanding. An approximating formula can be derived, losing a small amount of precision but giving a significant speed improvement. Furthermore, the formula is applicable to $n$--factor setting, not necessarily just two-factor. \\

\noindent Given an annuity 
\begin{equation*}
A(t) = A(t|{T_0},{T_N}) = \sum\limits_{i = 1}^N {{\tau _i}P(t,{T_i})},
\end{equation*}
payer swaption payoff can be expressed as
\begin{equation*}
V({T_0}) = A(t){\left( {S({T_0}) - K} \right)^ + },
\end{equation*}
where $S$ is the forward swap rate for the underlying swap
\begin{equation*}
S(t) = \frac{{P(t,{T_0}) - P(t,{T_N})}}{{A(t)}}.
\end{equation*}
It is well known that under the annuity measure ${\mathbb{Q}_A}$ associated with $A$ as numeraire, the forward swap rate is a ${\mathbb{Q}_A}$--martingale and we can write
\begin{equation*}
V(0) = A(0){\mathbb{E}^{{\mathbb{Q}_A}}}\left[ {{{\left( {S({T_0}) - K} \right)}^ + }} \right].
\end{equation*}
It can also be shown that $S(t)=S(t,x(t))$ follows
\begin{equation*}
dS(t) = q{(t,x(t))^{\rm{T}}}\sigma {(t)^{\rm{T}}}dW(t),
\end{equation*}
where $q$ is defined as
\begin{equation*}
{q_j}(t,x) = \frac{{P(t,{T_0},x){G_j}(t,{T_0}) - P(t,{T_N},x){G_j}(t,{T_N}) - S(t,x)\sum\limits_{i = 1}^N {{\tau _i}P(t,{T_i},x){G_j}(t,{T_i})} }}{{A(t,x)}}.
\end{equation*}
Despite $x$ being stochastic, ${q_j}(t,x(t))$ is very close to a constant and therefore as approximation we can write
\begin{equation*}
{q_j}(t,x(t)) \approx {q_j}(t,\bar x(t)),
\end{equation*}
where $\bar x(t)$ is a deterministic proxy for the random state variables $x(t)$. A reasonable proxy is to set $\bar x(t) = {\bf{0}}$ and will be used in our model.

Then the following Bachelier formula is a well-known result for normal models.
\begin{lemma}[Approximating swaption formula]
	Let $\bar x(t)$ be a deterministic function of time. Then the approximating payer swaption formula in the multifactor gaussian model reads
	\begin{equation*}
	V(0) = A(0)\left( {(S(0) - K)N(d) + \sqrt v {f_N}(d)} \right),
	\end{equation*}
	with 
	\begin{equation*}
	d = \frac{{S(0) - K}}{{\sqrt v }},v = \int\limits_0^{{T_0}} {{{\left\| {q{{(t,\bar x(t))}^{\rm{T}}}\sigma {{(t)}^{\rm{T}}}} \right\|}^2}} dt.
	\end{equation*}
\end{lemma}
Although this also involves integration, it is a much simpler integration than in the exact valuation formula \eqref{eq:swaption_formula_gaussian2f}.

\newpage

\bibliographystyle{plainnat}
\begin{thebibliography}{9}
	\bibitem{piterbargII}
	Andersen, Leif B., and Vladimir V. Piterbarg. 
	\textit{Interest rate modeling}. 
	London: Atlantic Financial Press, 2010. Print.
\end{thebibliography}

\end{document}
